%!TEX TS-program = xelatex
\documentclass[12pt, a4paper, oneside]{article}

\usepackage{amsmath,amsfonts,amssymb,amsthm,mathtools}  % пакеты для математики

\usepackage[english, russian]{babel} % выбор языка для документа
\usepackage[utf8]{inputenc} % задание utf8 кодировки исходного tex файла
\usepackage[X2,T2A]{fontenc}        % кодировка

\usepackage{fontspec}         % пакет для подгрузки шрифтов
\setmainfont{Linux Libertine O}   % задаёт основной шрифт документа

\usepackage{unicode-math}     % пакет для установки математического шрифта
\setmathfont[math-style=upright]{Neo Euler} % шрифт для математики

% Конкретный символ из конкретного шрифта
% \setmathfont[range=\int]{Neo Euler}

% Конкретный символ из конкретного шрифта
% \setmathfont[range=\int]{Neo Euler}

%%%%%%%%%% Работа с картинками %%%%%%%%%
\usepackage{graphicx}                  % Для вставки рисунков
\usepackage{graphics}
\graphicspath{{images/}{pictures/}}    % можно указать папки с картинками
\usepackage{wrapfig}                   % Обтекание рисунков и таблиц текстом

%%%%%%%%%%%%%%%%%%%%%%%% Графики и рисование %%%%%%%%%%%%%%%%%%%%%%%%%%%%%%%%%
\usepackage{tikz, pgfplots}  % язык для рисования графики из latex'a

%%%%%%%%%% Гиперссылки %%%%%%%%%%
\usepackage{xcolor}              % разные цвета

\usepackage{hyperref}
\hypersetup{
	unicode=true,           % позволяет использовать юникодные символы
	colorlinks=true,       	% true - цветные ссылки, false - ссылки в рамках
	urlcolor=blue,          % цвет ссылки на url
	linkcolor=red,          % внутренние ссылки
	citecolor=green,        % на библиографию
	pdfnewwindow=true,      % при щелчке в pdf на ссылку откроется новый pdf
	breaklinks              % если ссылка не умещается в одну строку, разбивать ли ее на две части?
}


\usepackage{todonotes} % для вставки в документ заметок о том, что осталось сделать
% \todo{Здесь надо коэффициенты исправить}
% \missingfigure{Здесь будет Последний день Помпеи}
% \listoftodos --- печатает все поставленные \todo'шки

\usepackage[paper=a4paper, top=20mm, bottom=15mm,left=20mm,right=15mm]{geometry}
\usepackage{indentfirst}       % установка отступа в первом абзаце главы

\usepackage{setspace}
\setstretch{1.15}  % Межстрочный интервал
\setlength{\parskip}{4mm}   % Расстояние между абзацами
% Разные длины в латехе https://en.wikibooks.org/wiki/LaTeX/Lengths


\usepackage{xcolor} % Enabling mixing colors and color's call by 'svgnames'

\definecolor{MyColor1}{rgb}{0.2,0.4,0.6} %mix personal color
\newcommand{\textb}{\color{Black} \usefont{OT1}{lmss}{m}{n}}
\newcommand{\blue}{\color{MyColor1} \usefont{OT1}{lmss}{m}{n}}
\newcommand{\blueb}{\color{MyColor1} \usefont{OT1}{lmss}{b}{n}}
\newcommand{\red}{\color{LightCoral} \usefont{OT1}{lmss}{m}{n}}
\newcommand{\green}{\color{Turquoise} \usefont{OT1}{lmss}{m}{n}}

\usepackage{titlesec}
\usepackage{sectsty}
%%%%%%%%%%%%%%%%%%%%%%%%
%set section/subsections HEADINGS font and color
\sectionfont{\color{MyColor1}}  % sets colour of sections
\subsectionfont{\color{MyColor1}}  % sets colour of sections

%set section enumerator to arabic number (see footnotes markings alternatives)
\renewcommand\thesection{\arabic{section}.} %define sections numbering
\renewcommand\thesubsection{\thesection\arabic{subsection}} %subsec.num.

%define new section style
\newcommand{\mysection}{
	\titleformat{\section} [runin] {\usefont{OT1}{lmss}{b}{n}\color{MyColor1}} 
	{\thesection} {3pt} {} } 


%	CAPTIONS
\usepackage{caption}
\usepackage{subcaption}
%%%%%%%%%%%%%%%%%%%%%%%%
\captionsetup[figure]{labelfont={color=Turquoise}}

\pagestyle{empty}


%%%%%%%%%% Свои команды %%%%%%%%%%
\usepackage{etoolbox}    % логические операторы для своих макросов

% Все свои команды лучше всего определять не по ходу текста, как это сделано в этом документе, а в преамбуле!

% Одно из применений - уничтожение какого-то куска текста!
\newbool{answers}
%\booltrue{answers}
\boolfalse{answers}

\usepackage{enumitem}
% бульпоинты в списках
\definecolor{myblue}{rgb}{0, 0.45, 0.70}
\newcommand*{\MyPoint}{\tikz \draw [baseline, fill=myblue,draw=blue] circle (2.5pt);}
\renewcommand{\labelitemi}{\MyPoint}

% расстояние в списках
\setlist[itemize]{parsep=0.4em,itemsep=0em,topsep=0ex}
\setlist[enumerate]{parsep=0.4em,itemsep=0em,topsep=0ex}

\begin{document}

\section*{Подготовка к кексу}

В течение одного из следущих семенаров вам предстоит решить кекс. То есть рассказать о том как бы вы с помощью методов машинного обучения попробовали бы решить жизненную проблему. Откуда бы вы брали данные, какие бы модели строили и тп. Эта небольшая pdf-ка призвана вам помочь немного привести своё кексовое мышление в порядок и вспомнить о каких жизненных ситуациях мы говорили на семинарах.  Несмотря на то, что по сравнению с предстоящим кексом, они были довольно мелки, в них были полезные мысли. 

Итак, на семинарах мы с вами прошлись по основным задачам машинного обучения и посмотрели на то, как они естественным образом возникают в маркетинге. Давайте тезисно вспомним о том, что это были за задачи в идеологическом плане. 

\subsection*{Кластеризация} 

\subsubsection*{Кекс про торговлю подарками}

Британский интернет-магазин подарков собрал данные по своим транзакциям за $2010-2011$ годы. В основном магазин работает с оптовиками.  Он хочет сегментировать своих клиентов по их характеристикам и более точечно проводить различные рекламные кампании. Нужно ему помочь! 

Наши рассуждения: 

\begin{itemize}
	\item Можно попробовать сегментировать клиентов стандартными приёмами, основанными на нашей экспертности. Например, по географическому положению, объёмам сделок, сделать RFM-сегментацию и тп.  
	
	\item Можно попробовать задействовать всю мощь машинного обучения! Например, в нашем датасете содержится огромное количество описаний товаров. Давайте обработаем эти тексты и на их основе проведём кластеризацию. Скорее всего, каждый оптовый покупатель концентрируется на каких-то определённых разновидностях подарков. Кластеризация по описаниям поможет нам выявить разновидности и настроить нашу рекламную кампанию, отталкиваясь от полученных по товарам кластеров. 
	
	\item \href{http://nbviewer.jupyter.org/github/FUlyankin/HSE_Data_Culture/blob/master/ML%20for%20marketing%202018%20autumn/sem_23/1.1%20Segmantation_clusterization.ipynb}{Подпробное описание задачи и код с картинками и решением} 
	
	\item  \textbf{Ещё немного необязательной информации.}  На паре мы выяснили, что кластеры, построенные по описаниям пересекаются. Это не удивительно, мы часто используем одни и те же слова для различных целей... К счастью есть более сложные модели, которые помогают учитывать наложение кластеров друг-на друга. Например про них и многое другое можно \href{http://nbviewer.jupyter.org/github/FUlyankin/ekanam_grand_research/blob/master/Posts/3.1%20Public_clusters.ipynb}{почитать вот тут.} Это \href{http://nbviewer.jupyter.org/github/FUlyankin/ekanam_grand_research/blob/master/Posts/0.%20Introduction.ipynb}{мой небольшой рисёрч.} Он получился довольно длинным, очень неформальным и слегка доставляющим. 
\end{itemize}


\subsubsection*{Кекс про рекламу}

Международное круизное агентство "Carnival Cruise Line" решило себя разрекламировать с помощью баннеров и обратилось для этого к вам. Чтобы протестировать, велика ли от таких баннеров польза, их будет размещено всего 20 штук по всему миру. Вам надо выбрать 20 таких локаций для размещения, чтобы польза была большой, и агентство продолжило с Вами сотрудничать.

Агентство крупное, и у него есть офисы по всему миру. Вблизи этих офисов оно и хочет разместить баннеры - легче договариваться и проверять результат. Также эти места должны хорошо просматриваться. 

Наши рассуждения: 

\begin{itemize}
	\item Банеры. Нужно, чтобы их чаще смотрели. В точках, где они стоят нужны большие скопления людей. Агенство круизное, значит нам нужны туристы. 
	
	\item Как найти большое скопление людей? По геолокации! Нужна база чекинов. 
	
	\item  Тогда мы сможем кластеризовать чекины, найти самые популярные места в окрестнсоти каждого офиса и поставить там банеры. Задача будет решена, на банеры будут смотреть, а нам дадут денег. 
	
	\item Ещё можно очистить данные от чекинов людей, которые точно не являются туристами. Например, можно посмотреть какие люди чекинились в районе в разные дни. Маловероятно, что турист будет чекиниться в одном и том же месте и в понедельник и в пятницу. Скорее всего, это местный житель. Эту идею, кстати говоря, придумала Ира :)  Можно придумать и другие улучшения! 

	\item \href{http://nbviewer.jupyter.org/github/FUlyankin/HSE_Data_Culture/blob/master/ML%20for%20marketing%202018%20autumn/sem_23/1.2%20Banners.ipynb}{Подпробное описание задачи и код с картинками и решением} 
\end{itemize}


\subsection*{Классификация} 

\subsubsection*{Кекс про приложение}

Мы хотим выпустить своё классное приложение под IOS, вот только беда в том, что мы не знаем как стать успешными. Срочно нужно что-нибудь придумать и разобраться с этой проблемой! 

Наши рассуждения: 

\begin{itemize}
	\item  На самом деле нам предстоит решать две задачи: на первых этапах нужно привлечь пользователей, а на дальнейших этапах нужно их не потерять. 
	
	\item Можно было бы собрать много-много примеров успешных приложений и провальных и внимательно изучить каждый конкретный случай. Тогда бы мы могли понять какие именно вещи стоит повторить, а какие повторять не стоит. Это хорошая идея и практика, однако мы тут занимаемся машинным обучением и хотим большего! 
	
	\item Можно попробовать обучить модель понимать какое приложение хорошее, а какое плохое. Тогда бы мы могли посмотреть на какие факторы ориентируется модель и при создании своего приложения обратить внимние именно на них. 
	
	\item Собираем данные с Appstore. Делим все прилодения на молодые (получил мало оценок) и старые (получили много оценок). Обучаем две логистические регрессии. Сравниваем какие факторы вносят положительный вклад на первых этапах развития приложения и на последующих. 
	
	\item \href{http://nbviewer.jupyter.org/github/FUlyankin/HSE_Data_Culture/blob/master/ML%20for%20marketing%202018%20autumn/sem_45/2.%20Classification_solution.ipynb}{Подпробное описание задачи и код с картинками и решением} Обратите внимание, что мы довольно грубо оценили нашу модель. В реальности с интерпретацией коэффициентов и их величин нужно быть очень аккуратным. Для того, чтобы выработать эту аккуратность,  в дополнение к машинке нужно заботать матстат и эконометрику.  
	
	\item Машинное обучение ни в коем случае не отменяет экспертный подход, в котором мы собираем успешные и неудачные кейсы. Эти подходы в данной задаче должны сочетать друг-друга. 
\end{itemize}


\subsubsection*{Регрессия} 

\subsubsection*{Кекс про продажи}

Walmart продаёт продукты в разных регионах США. Каждый магазин содержит несколько отделов. Магазин очень хочет спрогнозировать по каждому отделу для каждого магазина объём продаж. 

Зачем? Если мы привезли в магазин слишком мало товара, потребителем его не хватит. Мало того, что они не принесут нам денег, так ещё и станут к нам менее лояльными: "Не поедем в этот магазин. Там вечно ничего нет."

Если мы привезли в магазин слишком много товара, то возникают лишние расходы, связанные с хранением товаров, а также лишние расходы, связанные с просрочкой товаров.
Хотелось бы уметь избегать всех этих лишних расходов и привозить в каждый магазин ровно столько товара, сколько у нас купят. 

Ясное дело, что для разных типов товаров мы будем нести разные расходы на хранение, более того разные товары портятся с разной скоростью. В идеале было бы круто предсказывать продажи для каждой отдельной группы товаров.

Например, для овощей у нас одна модель, для телевизоров вторая, а функции потерь зависят от специфика каждого товара. На практике, скорее всего, так и делают. Мы только учимся и такое разнообразие задач нас угробит. Мы без детализации посмотрели на аггрегированную статистику Walmart. Мы оценивали регрессию, которая должна была бы предсказать продажи. 

При оценке модели мы поговорили о работе с выбросами, в данном случае это праздники. Мы боролись с праздниками. В них образуются акции по уценке товаров, из-за этого возникают всплески в продажах, то есть аномалии, то есть выбросы. Мы специфицировали нашу модель так, чтобы учесть праздники.  Выбросы довольно опасная штука. 

Кстати говоря, помнить про квантильную ошибку? Когда мы в прошлом семестре присваивали недопрогнозу больший вес, чем перепрогнозу? Тут можно попробовать использовать её. Потеря клиентов явно опаснее протухания продукта на складе. \href{http://nbviewer.jupyter.org/github/FUlyankin/HSE_Data_Culture/blob/master/ML%20for%20marketing%202018%20autumn/sem_67/3.1%20Продажи%20и%20линейная%20регрессия..ipynb}{Подпробное описание задачи и код с картинками и решением.} 


\subsection*{Другие полезные мелочи}

\subsection*{Ещё кексы} 

В Яндексе периодически проходят всякие крутые встречи, на которых люди делятся друг с другом опытом работы с данными. На одной из таких встреч обсуждали маркетинговые задачи. Рассказ о парочке маркетинговых кексов можно найти, например, вот в этой 30-минутной лекции: \url{ https://events.yandex.ru/lib/talks/6063/}.  

Возможно, вы почерпнёте из неё какие-то классные идеи, а потом используете их при решении своего кекса. 

\subsection*{О том как ритейл собирает данные о нас}

В кексе важно будет руководствоваться реалистичными предпосылками и источниками данных. Чтобы примерно представлять себе откуда ритейл берёт данные, здесь есть небольшой рассказ о них. В какой-то степени он дублирует лекцию выше. 

Для сбора данных и аналитики сайты используют специальные сервисы. Например, Яндекс.Метрику и Google Analytics. Эти сервисы позволяют анализировать то, что люди делают на сайте, с каких страниц они приходят, откуда они приходят (из поиска, с конкретного рекламного баннера и тп), какими демографическими характеристиками они обладают (пол, возраст, география и тп). 

Более того, для каждого пришедшего пользователя существуют очень разношёрстные данные о визитах: в какой последовательности он смотрел страницы, куда кликал мышью, как ей двигал и т.п.  Сервисы фактически показывает полную информацию о том, что происходит на сайте, позволяют выгрузить сырые данные и на их основе обучить какие-то модели. 

Предположим, что ЛОЛита два года назад зашла на сайт интернет-магазина сделать парочку покупок. Вчера она сделал это повторно. Как понять, что эти два захода принадлежат одному и тому же человеку? Обычно для этого используют систему из разных id. 

Если ЛОЛита заходил на сайт, используя свой личный кабинет, мы поймём, что это один и тот же человек по его внутреннему id. Другой способ идентифицировать человека --- использовать его аккаунт в google или яндексе.  Если человек зашёл на сайт, был залогинен в своей почте, и на сайте стояла метрика, мы сможем его отследить. 

Более слабым идентификатором является device-id устройства, с которого работает человек. Ясное дело, что сначала человек может зайти с телефона, потом с компьютера и, если он не залогинен, то система будет думать, что это два разных человека. 

Самым слабым идетификатором является id, построенный на основе куки человека. Если вы почистите в браузере куки, то этот id перезатрётся, и система будет думать, что вы новый человек.  Используя такую вложенную систему из адишников, мы можем понимать где выполнял действия один и тот же человек. 

В оффлайн-ритейле дело обстоит немного сложнее. Когда у магазина есть два чека в базе данных, он никак не может понять принадлежат они одному и тому же человеку или нет. Чтобы как-то исправить эту ситуацию и научиться агрегировать покупки, магазины придумывают всякие ухищрения, позволяющие им накопить данные. Например, систему бонусных карт, по id которых можно понять, что чеки принадлежат одному и тому же человеку. 

\end{document}



