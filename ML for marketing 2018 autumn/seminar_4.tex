%!TEX TS-program = xelatex
\documentclass[12pt, a4paper, oneside]{article}

\usepackage{amsmath,amsfonts,amssymb,amsthm,mathtools}  % пакеты для математики

\usepackage[english, russian]{babel} % выбор языка для документа
\usepackage[utf8]{inputenc} % задание utf8 кодировки исходного tex файла
\usepackage[X2,T2A]{fontenc}        % кодировка

\usepackage{fontspec}         % пакет для подгрузки шрифтов
\setmainfont{Linux Libertine O}   % задаёт основной шрифт документа

\usepackage{unicode-math}     % пакет для установки математического шрифта
\setmathfont[math-style=upright]{Neo Euler} % шрифт для математики

% Конкретный символ из конкретного шрифта
% \setmathfont[range=\int]{Neo Euler}


% Конкретный символ из конкретного шрифта
% \setmathfont[range=\int]{Neo Euler}

%%%%%%%%%% Работа с картинками %%%%%%%%%
\usepackage{graphicx}                  % Для вставки рисунков
\usepackage{graphics}
\graphicspath{{images/}{pictures/}}    % можно указать папки с картинками
\usepackage{wrapfig}                   % Обтекание рисунков и таблиц текстом

%%%%%%%%%%%%%%%%%%%%%%%% Графики и рисование %%%%%%%%%%%%%%%%%%%%%%%%%%%%%%%%%
\usepackage{tikz, pgfplots}  % язык для рисования графики из latex'a

%%%%%%%%%% Гиперссылки %%%%%%%%%%
\usepackage{xcolor}              % разные цвета

\usepackage{hyperref}
\hypersetup{
	unicode=true,           % позволяет использовать юникодные символы
	colorlinks=true,       	% true - цветные ссылки, false - ссылки в рамках
	urlcolor=blue,          % цвет ссылки на url
	linkcolor=red,          % внутренние ссылки
	citecolor=green,        % на библиографию
	pdfnewwindow=true,      % при щелчке в pdf на ссылку откроется новый pdf
	breaklinks              % если ссылка не умещается в одну строку, разбивать ли ее на две части?
}


\usepackage{todonotes} % для вставки в документ заметок о том, что осталось сделать
% \todo{Здесь надо коэффициенты исправить}
% \missingfigure{Здесь будет Последний день Помпеи}
% \listoftodos --- печатает все поставленные \todo'шки

\usepackage[paper=a4paper, top=20mm, bottom=15mm,left=20mm,right=15mm]{geometry}
\usepackage{indentfirst}       % установка отступа в первом абзаце главы

\usepackage{setspace}
\setstretch{1.15}  % Межстрочный интервал
\setlength{\parskip}{4mm}   % Расстояние между абзацами
% Разные длины в латехе https://en.wikibooks.org/wiki/LaTeX/Lengths


\usepackage{xcolor} % Enabling mixing colors and color's call by 'svgnames'

\definecolor{MyColor1}{rgb}{0.2,0.4,0.6} %mix personal color
\newcommand{\textb}{\color{Black} \usefont{OT1}{lmss}{m}{n}}
\newcommand{\blue}{\color{MyColor1} \usefont{OT1}{lmss}{m}{n}}
\newcommand{\blueb}{\color{MyColor1} \usefont{OT1}{lmss}{b}{n}}
\newcommand{\red}{\color{LightCoral} \usefont{OT1}{lmss}{m}{n}}
\newcommand{\green}{\color{Turquoise} \usefont{OT1}{lmss}{m}{n}}

\usepackage{titlesec}
\usepackage{sectsty}
%%%%%%%%%%%%%%%%%%%%%%%%
%set section/subsections HEADINGS font and color
\sectionfont{\color{MyColor1}}  % sets colour of sections
\subsectionfont{\color{MyColor1}}  % sets colour of sections

%set section enumerator to arabic number (see footnotes markings alternatives)
\renewcommand\thesection{\arabic{section}.} %define sections numbering
\renewcommand\thesubsection{\thesection\arabic{subsection}} %subsec.num.

%define new section style
\newcommand{\mysection}{
	\titleformat{\section} [runin] {\usefont{OT1}{lmss}{b}{n}\color{MyColor1}} 
	{\thesection} {3pt} {} } 


%	CAPTIONS
\usepackage{caption}
\usepackage{subcaption}
%%%%%%%%%%%%%%%%%%%%%%%%
\captionsetup[figure]{labelfont={color=Turquoise}}

\pagestyle{empty}

%%%%%%%%%% Свои команды %%%%%%%%%%
\usepackage{etoolbox}    % логические операторы для своих макросов

% Все свои команды лучше всего определять не по ходу текста, как это сделано в этом документе, а в преамбуле!

% Одно из применений - уничтожение какого-то куска текста!
\newbool{answers}
\booltrue{answers}
%\boolfalse{answers}

\usepackage{enumitem}
% бульпоинты в списках
\definecolor{myblue}{rgb}{0, 0.45, 0.70}
\newcommand*{\MyPoint}{\tikz \draw [baseline, fill=myblue,draw=blue] circle (2.5pt);}
\renewcommand{\labelitemi}{\MyPoint}

% расстояние в списках
\setlist[itemize]{parsep=0.4em,itemsep=0em,topsep=0ex}
\setlist[enumerate]{parsep=0.4em,itemsep=0em,topsep=0ex}

\begin{document}


\section*{Семинар 6-7: Удержание клиентов}

\subsection*{Задача 1}

	Бандерлог начинает все определения со слов «это доля правильных ответов»:
	\begin{enumerate}
		\item[а)] accuracy — это доля правильных ответов\ldots
		\item[б)] точность (precision) — это доля правильных ответов\ldots
		\item[в)] полнота (recall) — это доля правильных ответов\ldots
		\item[г)] TPR — это доля правильных ответов\ldots
	\end{enumerate}

\ifbool{answers}{
	\textbf{Решение:}
		
	Закончите определения Бандерлога так, чтобы они были, хм, правильными.
	
		\begin{enumerate}
			\item[а)] $\text{accuracy} = \frac{\text{TP} + \text{TN}}{\text{TP} +\text{FP} +\text{FN} +\text{TN}}$
			\item[б)] $\text{precision} = \frac{\text{TP}}{\text{TP} +\text{FP}}$
			\item[в)] $\text{recall} = \frac{\text{TP}}{\text{TP} +\text{FN}}$
			\item[г)] $\text{TPR} = \frac{\text{TP}}{\text{TP} +\text{FN}}$
		\end{enumerate}
}

\subsection*{Задача 2}

Бандерлог оценил три модели: нейросеть, случайный лес и KNN.  Он построил на тестовой выборке прогнозы и получил три матрицы ошибок: 

\begin{minipage}[t]{0.33\textwidth}
\begin{tabular}{|c|c|c|}
\hline
		              & $y=1$  &  $ y = 0$ \\  \hline 
$\hat y = 1$  &   $80$ &    $20$ \\      \hline 
$\hat y = 0$ &   $20$ &     $80$ \\      \hline 
\end{tabular}
\end{minipage}
\begin{minipage}[t]{0.33\textwidth}
\begin{tabular}{|c|c|c|}
	\hline
	& $y=1$  &  $ y = 0$ \\  \hline 
	$\hat y = 1$  &   $48$ &    $2$ \\      \hline 
	$\hat y = 0$ &   $52$ &     $98$ \\      \hline 
\end{tabular}
\end{minipage}
\begin{minipage}[t]{0.33\textwidth}
\begin{tabular}{|c|c|c|}
	\hline
	& $y=1$  &  $ y = 0$ \\  \hline 
	$\hat y = 1$  &   $10$ &    $20$ \\         \hline 
	$\hat y = 0$ &   $90$ &    $10000$ \\   \hline 
\end{tabular}
\end{minipage}





\begin{enumerate}
	\item[а)]  Найдите для всех трёх моделей долю правильных ответов. 
	\item[б)]   Найдите для всех трёх моделей точность (precision) и полноту (recall)
	\item[в)]   Несколько ситуаций, когда чё выбирать --- ? 
	\item[г)]  
\end{enumerate}

\ifbool{answers}{
	\textbf{Решение:}
	
	
	\begin{enumerate}
		\item[а)] 
		\item[б)] 
		\item[в)] 
		\item[г)]  
	\end{enumerate}
	
}



\subsection*{Задача 3}

	Бандерлог из Лога\footnote{деревня в Кадуйском районе Вологодской области} ведёт блог, любит считать логарифмы и оценивать логистические регрессии. С помощью нового алгоритма Бандерлог решил задачу классификации по трём наблюдениям и получил $b_i = \hat\P(y_i = 1|x_i)$.
	
\begin{center}
	\begin{tabular}{c|c}
		$y_i$ & $b_i$ \\
		\hline
		$1$  & $0.7$ \\
		$0$ & $0.2$ \\
		$0$ & $0.3$ \\
	\end{tabular}
\end{center}
	
	\begin{enumerate}
		\item Постройте ROC-кривую.
		\item Найдите площадь под ROC-кривой.
		\item Постройте PR-кривую (кривая точность-полнота).
		\item Найдите площадь под PR-кривой.
		\item Как по-английски будет «бревно»?
	\end{enumerate}

\ifbool{answers}{
	\textbf{Решение:}
	
}


\section{Ещё задачи!}

\subsection*{Задача 4}

Бандерлог обучил логистическую регрессию и получил вектор предсказанных вероятностей принадлежностей к классу $1$. 

$$p = \begin{bmatrix} 0.9 & 0.1 & 0.75 &  0.56 &  0.2 &  0.37 & 0.25 \end{bmatrix},$$

а вектор правильный ответов равен

$$y = \begin{bmatrix} 1 & 0 &0 & 1 &1 & 0 & 0. \end{bmatrix}$$

\begin{enumerate}
	\item[а)]  Бинаризуйте ответ по порогу $t$ и посчитайте точность и полноту для $t = 0.3$ и для  $t = 0.8$.
	\item[б)]  Две ситуации и когда какой юзать либо PR-кривая. 
	\item[в)]  Постройте ROC-кривую и найдите площадь под ней. 
 
\end{enumerate}

\ifbool{answers}{
	\textbf{Решение:}
	
	\begin{enumerate}
		\item[а)] 
		\item[б)] 
		\item[в)] 
	\end{enumerate}
	
}


\end{document}
